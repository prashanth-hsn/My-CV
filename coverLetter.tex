%!TEX TS-program = xelatex
%!TEX encoding = UTF-8 Unicode
% Awesome CV LaTeX Template for Cover Letter
%
% This template has been downloaded from:
% https://github.com/posquit0/Awesome-CV
%
% Authors:
% Claud D. Park <posquit0.bj@gmail.com>
% Lars Richter <mail@ayeks.de>
%
% Template license:
% CC BY-SA 4.0 (https://creativecommons.org/licenses/by-sa/4.0/)
%


%-------------------------------------------------------------------------------
% CONFIGURATIONS
%-------------------------------------------------------------------------------
% A4 paper size by default, use 'letterpaper' for US letter
\documentclass[11pt, a4paper]{awesome-cv}

% Configure page margins with geometry
\geometry{left=1.4cm, top=.8cm, right=1.4cm, bottom=1.8cm, footskip=.5cm}

% Specify the location of the included fonts
\fontdir[fonts/]

% Color for highlights
% Awesome Colors: awesome-emerald, awesome-skyblue, awesome-red, awesome-pink, awesome-orange
%                 awesome-nephritis, awesome-concrete, awesome-darknight
\colorlet{awesome}{awesome-darknight}
% Uncomment if you would like to specify your own color
% \definecolor{awesome}{HTML}{CA63A8}

% Colors for text
% Uncomment if you would like to specify your own color
% \definecolor{darktext}{HTML}{414141}
% \definecolor{text}{HTML}{333333}
% \definecolor{graytext}{HTML}{5D5D5D}
% \definecolor{lighttext}{HTML}{999999}
% \definecolor{sectiondivider}{HTML}{5D5D5D}

% Set false if you don't want to highlight section with awesome color
\setbool{acvSectionColorHighlight}{false}

% If you would like to change the social information separator from a pipe (|) to something else
\renewcommand{\acvHeaderSocialSep}{\quad\textbar\quad}


%-------------------------------------------------------------------------------
%	PERSONAL INFORMATION
%	Comment any of the lines below if they are not required
%-------------------------------------------------------------------------------
% Available options: circle|rectangle,edge/noedge,left/right
%\photo[circle,noedge,left]{./examples/profile}
\name{Prashanth}{Sheshappa}
\position{Software Engineer{\enskip\cdotp\enskip}C++ Expert}
\address{Hinter der Masch 26, 38114 Braunschweig}

\mobile{(+49) 17631424884}
\email{prashanth.hsn@gmail.com}

\linkedin{prashanth-sheshappa}

%-------------------------------------------------------------------------------
%	LETTER INFORMATION
%	All of the below lines must be filled out
%-------------------------------------------------------------------------------
% The company being applied to
\recipient
  {Company Recruitment Team}
  {Some company in Germany}
% The date on the letter, default is the date of compilation
\letterdate{\today}
% The title of the letter
\lettertitle{Job Application for Software Engineer}
% How the letter is opened
\letteropening{Dear Mr./Ms./Dr. LastName,}
% How the letter is closed
\letterclosing{Sincerely,}
% Any enclosures with the letter
\letterenclosure[Attached]{Resume}


%-------------------------------------------------------------------------------
\begin{document}

% Print the header with above personal information
% Give optional argument to change alignment(C: center, L: left, R: right)
\makecvheader[R]

% Print the footer with 3 arguments(<left>, <center>, <right>)
% Leave any of these blank if they are not needed
\makecvfooter
  {\today}
  {Prashanth Sheshappa ~~~·~~~ Cover Letter}
  {}

% Print the title with above letter information
\makelettertitle

%-------------------------------------------------------------------------------
%	LETTER CONTENT
%-------------------------------------------------------------------------------
\begin{cvletter}

% Entry: Write some good entry line.

I am a passionate C++ developer specialized in Scientific computing with over nine years of experience.
I found this job posted on the r/cpp subreddit and I was very thrilled at the prospect of working for Dassault Systemes, since
the CAE tools developed by Dassault Systemes are ubiquitous in the industry. With the combination of my C++  
and CAE software development experience, in addition to my Masters in Computational Sciences, I believe that I would make a great 
fit for the CFD Software Engineer position.

%Was kannst Du? (What can you do?)
At my current job with gns-mbh, I work on the development and maintenance of 
Animator4, an industry standard FEM Post-processor.  
Working on a software that is widely used in the industry has an advantage 
that the software we ship gets pushed to the limits and it keeps us on our toes 
to make sure that it is capable of handling everything that is thorwn at it. 
This experience has helped me gain an intuition to identify algorithms which 
can potentially cause performance issues and perform benchmarks to identify any 
bottlenecks before shipping new features.
I am the core maintainer for several input interfaces, which parse the FEM 
result data files. One of them is the Abaqus Odb reader interface. Working with 
the Abaqus ODB API has made me appreciate how easy it is to work with libraries 
that are well documented.
Over the past decade the trend with FEM anlysis is to analyse bigger geometry 
with finer meshes. Which means larger result files sometimes over 100GB. 
To achieve good import speeds and low memory footprints, I spend considerable time
in benchmarking the algorithms and data structures using tools such as gprof, 
intel-vtune, valgrind:massif etc. 

Observing the current trend in language feature development and modern computer 
hardwares, it is is more than obvious that writing single thread code today is 
a crime. To make use of all the modern hardware at our disposal, we have to 
write multithreaded code and use GPUs for general computing. Although it is 
difficult to move a 25 years old codebase to exploit the modern hardware, we 
are trying to write the new code using the modern approches offered by either 
the language itself or some special libraries. 
 
Apart from writing c++ code, I also take care of Nightly test result evaluation
and Official release coordination. This has helped me hone my planning and 
communication skills.

%Wer bist Du? (who are you?) 
I studied Masters in Computational Sciences at TU-Braunschweig with focus on 
Parallel computing, FEM and CFD. For my Master's thesis I upgraded an old 
Turbomachinery CFD post-processor from Fortran-77 to Fortran-95 and then 
parallelized using MPI Standards at the MAN Diesel and Turbo. This was a very 
successful undertaking and my advisor wrote me letter of recommendation, which
I have attached along with other relevant documents. 

%Warum willst Du gerade zu diesem Unternehmen? (Why do you want to join this company?)
%I have a lot of fun solving complex algorithmic and architecural problems. I appreciate the dynamic nature of the software engineering. from the programming language to the tooling, everything keeps evolving at a rapid pace. For example since 2011, C++ standards commitee is churning out major updates every three years. It is exciting to try new feature of the language and witness the direction in which computing is headed.



%Was hat das Unternehmen davon, wenn es Dich einstellt? (What does the company benefit from if it hires you
I throughly enjoy solving challenging problems with designing algorithms and 
data structures. 
I believe good Software takes more than just good engineers. It takes a team with good communication,
 honest critique, fair appreciation and open culture.


\end{cvletter}


%-------------------------------------------------------------------------------
% Print the signature and enclosures with above letter information
\makeletterclosing

\end{document}
